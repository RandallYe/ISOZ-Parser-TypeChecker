\begin{zsection}
\SECTION cross \parents function\_toolkit
\end{zsection}

\begin{zed}
[NAME, ID, DEPARTMENT]
\end{zed}

%%%Zinword \op9 op9 
%%%Zinword \op5 op5 
%\begin{zed}
%% an operator which binds more tightly than cross
%\generic 9 \leftassoc (\_ \op9 \_) 
%\end{zed}
%
%\begin{gendef}[X]
%\_ \op9 \_ : \power X \cross \power X \fun \power X
%\end{gendef}
%
%\begin{zed}
%\generic 5 \leftassoc (\_ \op5 \_) 
%\end{zed}
%
%\begin{gendef}[X]
%\_ \op5 \_ : \power X \cross \power X \fun \power X
%\end{gendef}

\begin{zed}
EMPLOYEE == ID \cross NAME \cross DEPARTMENT \\
\end{zed}
\begin{zed}
EMPLOYEE1 == (ID \cross NAME) \cross DEPARTMENT \\
\end{zed}
\begin{zed}
EMPLOYEE2 == ID \cross (NAME \cross DEPARTMENT) \\
\end{zed}
\begin{zed}
% the precedence of \cup is 30
EMPLOYEE3 == ID \cross NAME \cup NAME \cross DEPARTMENT \\
\end{zed}
\begin{zed}
% the precedence of \rel is 5
EMPLOYEE4 == ID \cross NAME \rel NAME \cross DEPARTMENT \\
\end{zed}
\begin{zed}
% the precedence of \cup is 30
EMPLOYEE5 == ID \cross (NAME \cup NAME \cross DEPARTMENT) \\
\end{zed}
\begin{zed}
% the precedence of \rel is 5
EMPLOYEE6 == (ID \cross NAME \rel NAME) \cross DEPARTMENT \\
\end{zed}
