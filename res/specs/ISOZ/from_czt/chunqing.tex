%\documentstyle[12pt,coz,times]{report}
\documentclass{llncs}
\usepackage{pt}
\usepackage{units}
\usepackage{coz}
\usepackage{epsf}
\usepackage{lncsexample}
\conttrue
\def\kph{\km \mulnuit{\hour}{-1}}

\begin{document}

\begin{zsection}
\SECTION golf \parents standard\_toolkit
\end{zsection}

Basically, the specification is for a golf tournament. It has a set of players with their corresponding
scores. New players can be assigned while it can delete a player in the list. Also, it can calculate
the score for each player based on their own shots.


Using Name as a set to present players.
\begin{zed}
[Name]
\end{zed}
Using an axiomatic definition to limit the number of players in the tournament.
\begin{axdef}
maxPlayer : \nat
\where maxPlayer \geq 100
\end{axdef}
This is the state schema for the tournament. It contains a set of players, with a full function
named "score" to calculate the grade of a specific player.
\begin{schema}{Tournament}
players : \power Name \\
score : Name \pfun \num
\where \# players \leq maxPlayer \\
\dom score = players
\end{schema}

Intially, the set of players is empty.
\begin{schema}{TournamentInit}
Tournament
\where players = \emptyset
\end{schema}

The "Newplayer" schema can add a new player to the set of players.
\begin{schema}{NewPlayer}
\Delta Tournament \\
p? : Name
\where p? \notin players \\
players' = players \cup \{p?\} \\
score' = score \cup \{ p? \mapsto 0 \}
\end{schema}

The "PlayerResigns" schema can resign a player from the set of players.
\begin{schema}{PlayerResigns}
\Delta Tournament \\
p? : Name
\where p? \in players\\
players' = players \setminus \{p?\} \\
score' = \{p?\} \ndres score
\end{schema}

The operation schema "PlayHole" can update the player's score based on the shots he/she has
\begin{schema}{PlayHole}
\Delta Tournament \\
p? : Name \\
shots? : \num
\where p? \in players \\
players' = players \\
score' = score \oplus \{ p? \mapsto (score(p?) + shots?) \}
\end{schema}



\begin{zsection}
\SECTION enterprise \parents standard\_toolkit
\end{zsection}

\begin{zed}
%[NAME]
NAME ::= chen | chun 
\end{zed}
\begin{zed}
ID == \nat
\end{zed}
\begin{zed}
DEPARTMENT ::= administration | manufacturing | research
\end{zed}
\begin{zed}
EMPLOYEE == ID \cross NAME \cross DEPARTMENT
\end{zed}
\begin{axdef}
  Frank, Aki : EMPLOYEE
\where Frank = (12, chen, administration) \\
  Aki = (24, chun, research)
\end{axdef}

\begin{zsection}
\SECTION editor \parents standard\_toolkit
\end{zsection}

\begin{zed}
[CHAR]
\end{zed}
\begin{zed}
TEXT == \seq CHAR
\end{zed}
\begin{axdef}
maxsize: \nat
\where maxsize \leq 65535
\end{axdef}
\begin{schema}{Editor}
left, right : TEXT
\where \#(left \cat right) \leq maxsize
\end{schema}
\begin{schema}{Init}
Editor
\where left=right=\langle \rangle
\end{schema}
\begin{axdef}
printing: \power CHAR
\end{axdef}
\begin{schema}{Insert}
\Delta Editor \\
ch? : CHAR
\where ch? \in printing \\
left'= left \cat \langle ch? \rangle\\
right'= right
\end{schema}
\begin{axdef}
right\_arrow: CHAR 
\where right\_arrow \notin printing
\end{axdef}
\begin{schema}{Forward}
\Delta Editor \\
ch? : CHAR 
\where ch? = right\_arrow \\
left'= left \cat \langle head (right) \rangle \\
right'= tail (right)
\end{schema}
\begin{schema}{EOF}
Editor 
\where right = \langle \rangle
\end{schema}
\begin{schema}{RightArrow}
ch? : CHAR 
\where ch? = right\_arrow
\end{schema}
\begin{zed}
T\_Forward == Forward \lor (EOF \land RightArrow \land \Xi
Editor)
\end{zed}



\begin{zsection}
\SECTION console \parents standard\_toolkit
\end{zsection}

\begin{zed}
[EVENT]
\end{zed}

\begin{zed}
[DISPLAY, SETTING, VALUE, CHAR]
\end{zed}

\begin{zed}
TEXT == \seq CHAR
\end{zed}

\begin{zed}
MODE ::= idle | dialog
\end{zed}

\begin{schema}{Console}
display: DISPLAY \\
mode: MODE \\
buffer: TEXT \\
setting: SETTING
\end{schema}

\begin{schema}{Event}
\Delta Console\\
e? : EVENT
\end{schema}
\begin{zed}
Ignore == Event \land \Xi Console
\end{zed}

\begin{zed}
BUTTON ::= accept | cancel
\end{zed}

\begin{axdef}
disp: EVENT \pfun DISPLAY
\end{axdef}
\begin{schema}{SelectDisplay}
Event 
\where mode = idle \\
e? \in \dom disp \\
display' = disp (e?) \\
mode' = mode
\end{schema}
\begin{schema}{IgnoreDisplay}
Ignore 
\where e? \in \dom disp\\
mode = dialog
\end{schema}
\begin{zed}
DisplayEvent == SelectDisplay \lor IgnoreDisplay
\end{zed}

\begin{axdef}
stg : EVENT \pfun SETTING
\end{axdef}
\begin{schema}{SelectSetting}
Event 
\where mode = idle \\
e? \in \dom stg \\
setting' = stg(e?) \\
mode' = dialog \\
buffer' = \emptyset \\
display' = display
\end{schema}
\begin{schema}{IgnoreSetting}
Ignore 
\where e? \in \dom stg \\
mode = dialog
\end{schema}
\begin{zed}
SettingEvent == SelectSetting \lor IgnoreSetting
\end{zed}

\begin{axdef}
char : EVENT \pfun CHAR \\
edit : (TEXT \cross CHAR) \fun TEXT
\end{axdef}
\begin{schema}{GetChar}
Event 
\where mode = dialog \\
e? \in \dom char \\
buffer' = edit (buffer, char~e?) \\
mode' = mode \\
setting' = setting \\
display' = display
\end{schema}
\begin{schema}{IgnoreChar}
Ignore 
\where e? \in \dom char \\
mode = idle
\end{schema}
\begin{zed}
CharEvent == GetChar \lor IgnoreChar
\end{zed}

\begin{axdef}
button : EVENT \pfun BUTTON \\
value : TEXT \fun VALUE \\
valid : SETTING \rel VALUE
\end{axdef}
\begin{schema}{Accept}
Event 
\where mode = dialog \\
e? \in \dom button \\
button(e?)= accept \\
value(buffer) = valid(setting)\\
mode' = idle \\
display' = display
\end{schema}
\begin{schema}{Reprompt}
Event 
\where mode = dialog \\
e? \in \dom button \\
button(e?) = accept \\
\lnot valid(setting) = value(buffer)\\
buffer' = buffer \\
mode' = mode \\
display' = display
\end{schema}
\begin{schema}{Cancel}
Event 
\where mode = dialog \\
e? \in \dom button \\
button(e?) = cancel \\
mode' = idle \\
display' = display
\end{schema}
\begin{zed}
ButtonEvent == Accept \lor Cancel \lor Reprompt
\end{zed}

\begin{schema}{Machine}
measure, prescribed : SETTING \fun VALUE
\end{schema}
\begin{schema}{NewSetting}
\Delta Machine \\
s? : SETTING; v? : VALUE 
\where prescribed' = prescribed \oplus \{s? \mapsto v? \}\\
measure' = measure 
\end{schema}
\begin{schema}{Sys}
Console \\
Machine
\end{schema}
\begin{schema}{ChangeSetting}
\Delta Sys \\
Accept \\
NewSetting
\where s? = setting \\
v? = value(buffer)
\end{schema}
\begin{zed}
SysDisplayEvent == DisplayEvent \land \Xi Machine
\end{zed}
\begin{zed}
SysSettingEvent == SettingEvent \land \Xi Machine
\end{zed}
\begin{zed}
SysCharEvent == CharEvent \land \Xi Machine
\end{zed}
\begin{zed}
SysButtonEvent == ChangeSetting \lor (Reprompt \land \Xi
Machine) \lor (Cancel \land \Xi Machine)
\end{zed}



\begin{zsection}
\SECTION queens \parents standard\_toolkit
\end{zsection}

\begin{zed}
SIZE == 8
\end{zed}

\begin{zed}
FILE == 1 \upto SIZE
\end{zed}

\begin{zed}
RANK == 1 \upto SIZE
\end{zed}

\begin{zed}
SQUARE == FILE \cross RANK
\end{zed}

\begin{zed}
DIAGONAL == 1 - SIZE \upto 2 * SIZE
\end{zed}

\begin{axdef}
up, down: SQUARE \fun DIAGONAL
\where \forall f : FILE; r : RANK @ up(f, r) = r - f \land down(f, r)
= r + f
\end{axdef}

\begin{schema}{Queens}
squares : FILE \bij RANK 
\where \{squares \dres up, squares \dres down \} \subseteq SQUARE
\inj DIAGONAL
\end{schema}



\begin{zsection}
\SECTION documents \parents standard\_toolkit
\end{zsection}

\begin{zed}
[PERSON, DOCUMENT]
\end{zed}

\begin{axdef}
permission: DOCUMENT \rel PERSON
\end{axdef}

\begin{axdef}
doug, aki, phil : PERSON \\
spec, design, code : DOCUMENT
\end{axdef}

\begin{schema}{Documents}
check\_out: DOCUMENT \pfun PERSON
\where
check\_out \subseteq permission
\end{schema}

\begin{schema}{CheckOut}
\Delta Documents \\
p? : PERSON \\
d? : DOCUMENT
\where
d? \notin \dom check\_out \\
(d?, p?) \in permission \\
check\_out' = check\_out \cup \{(d?, p?)\}
\end{schema}

\begin{zed}
CheckedOut == [\Xi Documents; d? : DOCUMENT | d? \in \dom
check\_out]
\end{zed}

\begin{schema}{Unauthorized}
\Xi Documents \\
p? : PERSON \\
d? : DOCUMENT 
\where (d?, p?) \notin permission
\end{schema}

\begin{zed}
T\_CheckOut == CheckOut \lor CheckedOut \lor Unauthorized
\end{zed}



\begin{zsection}
\SECTION birthday\_book \parents standard\_toolkit
\end{zsection}

\begin{zed}
    [NAME, DATE] 
\end{zed}
\begin{schema}{BirthdayBook}
known: \power NAME \\
birthday : NAME \pfun DATE
\where
known = \dom birthday
\end{schema}
\begin{schema}{BirthdayBook1}
names : \nat \fun NAME \\
dates: \nat \fun DATE \\
hwm: \nat
\where
\forall i, j : 1 \upto hwm @ i \neq j \implies names(i) \neq names(j)
\end{schema}
\begin{schema}{Abs}
BirthdayBook \\
BirthdayBook1
\where
known = \{i : 1 \upto hwm @ names(i) \} \\
\forall i : 1 \upto hwm @ birthday(names(i)) = dates(i)
\end{schema}

\begin{schema}{AddBirthday1}
\Delta BirthdayBook1 \\
name? : NAME \\
date? : DATE
\where
\forall i : 1 \upto hwm @ name? \neq names(i) \\
hwm' = hwm + 1 \\
names' = names \oplus \{ hwm' \mapsto name? \} \\
dates' = dates \oplus \{ hwm' \mapsto date? \}
\end{schema}

\begin{schema}{FindBirthday1}
\Xi BirthdayBook1 \\
name? : NAME \\
date! : DATE
\where
\exists i: 1 \upto hwm @ name? = names(i) \land date! = dates(i)
\end{schema}

\begin{schema}{AbsCards}
cards : \power NAME \\
cardlist : \nat \fun NAME \\
ncards : \nat
\where
cards = \{ i : 1 \upto ncards @ cardlist(i) \}
\end{schema}

\begin{schema}{Remind1}
\Xi BirthdayBook1 \\
today? : DATE \\
cardlist! : \nat \fun NAME \\
ncards! : \nat
\where
\{ i : 1 \upto ncards! @ cardlist!(i) \} = \{ j : 1 \upto hwm | dates(j) = today? @ names(j) \}
\end{schema}
\begin{schema}{InitBirthdayBook1}
BirthdayBook1
\where
hwm = 0
\end{schema}



\begin{zsection}
\SECTION access \parents standard\_toolkit
\end{zsection}

\begin{zed}
[ADDR, PAGE]
\end{zed}

\begin{zed}
DATABASE == ADDR \fun PAGE
\end{zed}

\begin{schema}{CheckSys}
working: DATABASE \\
backup: DATABASE
\end{schema}

\begin{schema}{Access}
\Xi CheckSys \\
a? : ADDR \\
p! : PAGE
\where
p! = working(a?)
\end{schema}

\begin{schema}{Update}
\Delta CheckSys \\
a? : ADDR \\
p? : PAGE
\where
working' = working \oplus \{a? \mapsto p? \} \\
backup' = backup
\end{schema}

\begin{schema}{CheckPoint}
\Delta CheckSys
\where
working' = working \\
backup' = working
\end{schema}

\begin{schema}{Restart}
\Delta CheckSys
\where
working' = backup \\
backup' = backup
\end{schema}

\begin{schema}{Master}
master : DATABASE
\end{schema}

\begin{schema}{Changes}
changes : ADDR \pfun PAGE
\end{schema}

\begin{schema}{CheckSys1}
Master \\
Changes
\end{schema}

\begin{schema}{Abs}
CheckSys \\
CheckSys1
\where
backup = master \\
working = master \oplus changes
\end{schema}

\begin{schema}{ReadMaster}
\Xi Master \\
a? : ADDR \\
p! : PAGE
\where
p! = master(a?)
\end{schema}

\begin{schema}{Update1}
\Delta CheckSys1 \\
a? : ADDR \\
p? : PAGE
\where
master' = master \\
changes' = changes \oplus \{a? \mapsto p? \}
\end{schema}

\begin{schema}{CheckPoint1}
\Delta CheckSys1
\where
master' = master \oplus changes \\
changes' = \emptyset
\end{schema}

\begin{schema}{Restart1}
\Delta CheckSys1
\where master' = master \\
changes' = \emptyset
\end{schema}



\begin{zsection}
\SECTION pool \parents standard\_toolkit
\end{zsection}

\begin{schema}{Gimel}
y, z : \num \\
z : 1 \upto 10
\where
1 \leq z \leq 10 \\
y = z * z
\end{schema}
\begin{schema}{Aleph}
x, y : \num
\where
x < y
\end{schema}
\begin{axdef}
limit : \nat
\where
limit \leq 65535
\end{axdef}

\begin{schema}{Counter}
value : \nat
\where
value \leq limit
\end{schema}

\begin{genschema}{Pool}{RESOURCE}
owner: RESOURCE \pfun USER \\
free : \power RESOURCE
\where
(\dom owner) \uni free = RESOURCE \\
(\dom owner) \int free = \emptyset
\end{genschema}



\begin{zsection}
\SECTION something \parents standard\_toolkit
\end{zsection}

\begin{zed}
[FIELD, SETTING]
\end{zed}
\begin{zed}
VALUE == \num
\end{zed}
\begin{zed}
SETUP == SETTING \fun VALUE
\end{zed}
\begin{axdef}
prescription : FIELD \pfun SETUP
\end{axdef}

\begin{zed}
  abs == \lambda x:\num @ max \{ x, \negate x \}
\end{zed}

\begin{schema}{Field}
field: FIELD \\
measured, prescribed: SETUP \\
overridden : SETTING \pfun VALUE
\where field \in \dom prescription \\
prescribed = prescription(field)
\end{schema}

\begin{zed}
[INTERLOCK]
\end{zed}
\begin{zed}
INTLK ::= clear | set
\end{zed}
\begin{zed}
READY ::= ready | not\_ready | override
\end{zed}

\begin{schema}{Intlk}
therapy\_intlk : INTLK \\
intlk : INTERLOCK \fun INTLK \\
status : SETTING \fun READY
\end{schema}

\begin{axdef}
safe : \power SETUP \\
match : SETUP \rel SETUP
\end{axdef}

%\begin{schema}{SafeTreatment}
%measured, prescribed : SETUP 
%\where measured \in safe \\
%match(measured) = prescribed \\
%prescribed \in \ran prescription
%\end{schema}

\begin{axdef}
selection, scale : \power SETTING
\end{axdef}

\begin{axdef}
tol : scale \fun VALUE \\
valid : SETTING \fun \power VALUE
\end{axdef}

\begin{zed}
  \relation ( Match \varg )
\end{zed}

\begin{zed}
  \relation ( Safe \varg )
\end{zed}

\begin{zed}
  \relation ( Overridden \varg )
\end{zed}

\begin{zed}
  \relation ( Ready \varg )
\end{zed}

\begin{axdef}
Match \varg : \power(SETTING \cross SETUP \cross Field) \\
Safe \varg, Overridden \varg, Ready \varg : SETTING \rel Field 
\where \forall s : SETTING; setup : SETUP; Field @ \\
\t1 (Safe(s, \theta Field) \iff measured (s) \in valid(s)) \land
\also
\t1 (Match(s, setup, \theta Field) \iff (s \in selection \land
measured(s) = setup(s) \lor \\
\t1 \t1 (s \in scale \land abs(measured(s) - setup (s)) \leq
tol(s))))\land
\also
\t1 (Overridden(s, \theta Field) \iff s \in \dom overridden \land
Safe(s, \theta Field) \land \\
\t1 \t1 Match(s, overridden, \theta Field)) \land
\also
\t1 (Ready(s, \theta Field) \iff Safe(s, \theta Field) \land
Match(s, prescribed, \theta Field))
\end{axdef}

\begin{schema}{TreatmentStatus}
Field \\
Intlk \\
\where status = \\
\t1 (\lambda s: SETTING @ not\_ready) \oplus \\
\t1 (\lambda s: SETTING | Overridden(s, \theta Field)@ override)
\oplus \\
\t1 (\lambda s: SETTING | Ready(s, \theta Field) @ ready)
\end{schema}

\begin{schema}{SafeTreatment}
TreatmentStatus \\
\where \ran intlk = \{clear\} \\
\ran status \subseteq \{ready, override\}
\end{schema}

\begin{schema}{ScanIntlk}
\Xi Field \\
\Delta Intlk \\
\where TreatmentStatus~' \\
therapy\_intlk' = if \, SafeTreatment~' \, then \, clear \, else \, set
\end{schema}

\end{document}
